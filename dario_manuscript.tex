1era pagina:

such as the PARSEC, for example ... (falta la coma)

Next, a careful optimization of the potential... (el "a" dejalo)

The calculations ARE constrained to Hamiltonians...

If the fist order fails, it would not have ... (saca el indeed)

no pongas como sophisticated el time dependent.
Pone to more sophisticated fully quantum mechanical approaches
y ahi tirale un TDCC, un R-matrix y un CCC

Esta mal lo que decis del CDW-EIS.
Cuando hablas de HF, ahi podes decir que eso fue comprobado con
los calculos de CDW-EIS

No veo cual es la ventaja de complicar todo reemplazando PP por PPA.
Por lo menos no hagas eso en las formulas! o sea, V_PP no le pongas V_PPA

no se escribe "behaves as -1/r " se pone -r^-1


4.2 Methane: we considered CH4, which is a highly symmetric molecule and
therefore, can be described with an angular averaged potential

en lugar de previous works... pone cuales son los orbitales que incluye la
base. Luego deci que higher l orbitals, needed to account further polarisation
effects, produces changes in the energies of about xxxx. However, in order
to isolate the effects of the basis .... same footing.


4.3 Collisional processes:

The orientation is important (no es crucial!).
The orientation of the molecular targets are important for determining  the cross
sections of collisional processes. However, it is generally not  pre-established
in the experiments.

referencias:
sacale el accesed el 18 de enero a la pagina web.

agregale al TDCC la referencia:

M.S. Pindzola, F. Robicheaux, S.D. Loch, J.C. Berengut, T. Topcu, J.  Colgan, M. Foster, D.C. Griffin, C.P. Ballance, D.R. Schultz, T.  Minami, N.R. Badnell, M.C. Witthoeft, D.R. Plante, D.M. Mitnik, J.A.  Ludlow and U. Kleiman, ``The time-dependent close-coupling method for  atomic and molecular collision processes", J. Phys. B 40 , R39-R60  (2007).




