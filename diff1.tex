\documentclass[10pt]{article}
%DIF LATEXDIFF DIFFERENCE FILE
%DIF DEL dimmolv0.tex   Mon Apr  1 16:03:13 2019
%DIF ADD dimmol.tex     Fri Apr  5 17:12:25 2019
\usepackage{amsmath,amssymb}
\usepackage{graphicx}
\usepackage{authblk}
% \usepackage[square,sort,comma,numbers]{natbib}
\usepackage{float}
\usepackage{xcolor}
\usepackage[margin=1.1in]{geometry}
\usepackage{hyperref}


\hyphenation{pseudo-charges pseudo-potential mo-men-tum lith-ium}
%%%%%%%%%%%%%%%%%%%%%%%%%%%%%%%%%%%%%%%%%%%%%%%%%%%%%%%%%%%%%%%%%%%%%
%%%%%%%%%%%%%%%%%%%%%%%%%%%%%%%%%%%%%%%%%%%%%%%%%%%%%%%%%%%%%%%%%%%%%
%DIF PREAMBLE EXTENSION ADDED BY LATEXDIFF
%DIF UNDERLINE PREAMBLE %DIF PREAMBLE
\RequirePackage[normalem]{ulem} %DIF PREAMBLE
\RequirePackage{color}\definecolor{RED}{rgb}{1,0,0}\definecolor{BLUE}{rgb}{0,0,1} %DIF PREAMBLE
\providecommand{\DIFaddtex}[1]{{\protect\color{blue}\uwave{#1}}} %DIF PREAMBLE
\providecommand{\DIFdeltex}[1]{{\protect\color{red}\sout{#1}}}                      %DIF PREAMBLE
%DIF SAFE PREAMBLE %DIF PREAMBLE
\providecommand{\DIFaddbegin}{} %DIF PREAMBLE
\providecommand{\DIFaddend}{} %DIF PREAMBLE
\providecommand{\DIFdelbegin}{} %DIF PREAMBLE
\providecommand{\DIFdelend}{} %DIF PREAMBLE
%DIF FLOATSAFE PREAMBLE %DIF PREAMBLE
\providecommand{\DIFaddFL}[1]{\DIFadd{#1}} %DIF PREAMBLE
\providecommand{\DIFdelFL}[1]{\DIFdel{#1}} %DIF PREAMBLE
\providecommand{\DIFaddbeginFL}{} %DIF PREAMBLE
\providecommand{\DIFaddendFL}{} %DIF PREAMBLE
\providecommand{\DIFdelbeginFL}{} %DIF PREAMBLE
\providecommand{\DIFdelendFL}{} %DIF PREAMBLE
%DIF END PREAMBLE EXTENSION ADDED BY LATEXDIFF
%DIF PREAMBLE EXTENSION ADDED BY LATEXDIFF
%DIF HYPERREF PREAMBLE %DIF PREAMBLE
\providecommand{\DIFadd}[1]{\texorpdfstring{\DIFaddtex{#1}}{#1}} %DIF PREAMBLE
\providecommand{\DIFdel}[1]{\texorpdfstring{\DIFdeltex{#1}}{}} %DIF PREAMBLE
%DIF END PREAMBLE EXTENSION ADDED BY LATEXDIFF

\begin{document}

\title{Collision processes in atoms and molecules using effective 
potentials}
\author[ ]{Alejandra M.P. Mendez}
\author[ ]{Dar\'io M. Mitnik}
\author[ ]{Jorge E. Miraglia}
\affil[ ]{\small
Instituto de Astronom\'ia y F\'isica del Espacio,  
Universidad de Buenos Aires -- 
Consejo Nacional de Investigaciones Cient\'ificas y T\'ecnicas,
Buenos Aires, Argentina}
% \affil[2]{Departamento de F\'isica, Facultad de Ciencias Exactas y Naturales, 
% Universidad de Buenos Aires, Buenos Aires, Argentina}

\maketitle

% \scriptsize
\tableofcontents

% \normalsize
\begin{abstract}
We investigate the feasibility of using pseudopotentials to generate
the bound and continuum orbitals needed in collisional calculations. 
By examination of several \DIFaddbegin \DIFadd{first--order }\DIFaddend inelastic processes, we demonstrate 
the inconveniences of this approximation. Instead, we prescribe 
the usage of effective potentials obtained with the \DIFdelbegin \DIFdel{Depurated Inversion 
Method }\DIFdelend \DIFaddbegin \DIFadd{depurated inversion 
method }\DIFaddend (DIM), which is described in this work. We also extended this method 
for molecular systems. Calculations of single photoionisation and 
proton--impact ionisation with DIM show good agreements with 
experimental results.
\end{abstract}


%%%%%%%%%%%%%%%%%%%%%%%%%%%%%%%%%%%%%%%%%%%%%%%%%%%%%%%%%%%%%%%%%%%%%
\section{Outline}

Inelastic transition calculations require the representation of 
the bound and continuum states involved in the collisional processes. 
The hypothetical existence of an effective one--electron local 
potential accounting for these states would allow \DIFdelbegin \DIFdel{to generate in a 
simpler way }\DIFdelend \DIFaddbegin \DIFadd{generating more directly
}\DIFaddend the orthogonal wavefunctions for the interacting particles.
This \DIFdelbegin \DIFdel{should include detailed }\DIFdelend \DIFaddbegin \DIFadd{approach should include individual }\DIFaddend $nl$--orbital potentials, a feature 
missing in most of the standard density functional methods. 
The idea of replacing a many--body, nonlocal interaction by
an effective one--electron equation opens up the possibility of studying
extremely complex systems with high accuracy. 
However, this is not a simple task.

In this context, one \DIFdelbegin \DIFdel{clever idea emerge }\DIFdelend \DIFaddbegin \DIFadd{bright idea emerges }\DIFaddend from the pseudopotential
approximation, in which all the complexity of the wavefunctions near 
the core \DIFdelbegin \DIFdel{-- that normally }\DIFdelend \DIFaddbegin \DIFadd{--that usually }\DIFaddend consumes a huge numerical \DIFdelbegin \DIFdel{effort --, }\DIFdelend \DIFaddbegin \DIFadd{effort-- }\DIFaddend is avoided.
For instance, density functional theory codes using \DIFdelbegin \DIFdel{pesudopotentials}\DIFdelend \DIFaddbegin \DIFadd{pseudopotentials}\DIFaddend , 
such as the {\sc parsec} for example \cite{parsecwebpage,Chelikowsky1994}, 
permit to use an equally--spaced grid involving a relatively small 
number of points. 
Otherwise, the use of realistic potentials describing \DIFdelbegin \DIFdel{properly }\DIFdelend the nucleus 
Coulomb potential \DIFdelbegin \DIFdel{, }\DIFdelend requires a high density of points concentrated at the 
origin to describe \DIFdelbegin \DIFdel{precisely }\DIFdelend what the pseudopotentials cast aside. 
Thus, if this approach is applicable in the field of collisions theory, 
one would save \DIFaddbegin \DIFadd{an }\DIFaddend enormous amount of computational resources.
In this article, we explore the possibility of using pseudopotentials 
within the single electron model to calculate inelastic transitions. 
Single \DIFdelbegin \DIFdel{photoionization}\DIFdelend \DIFaddbegin \DIFadd{photoionisation}\DIFaddend , excitation, \DIFdelbegin \DIFdel{ionization }\DIFdelend \DIFaddbegin \DIFadd{ionisation }\DIFaddend and electron capture were
calculated in first perturbative order. We found two \DIFdelbegin \DIFdel{important drawbacks 
in this }\DIFdelend \DIFaddbegin \DIFadd{essential drawbacks 
in these }\DIFaddend applications. First, the range of validity is restrained to 
\DIFdelbegin \DIFdel{very
}\DIFdelend small momentum transfers.
Second, the internal information of the wavefunctions can play a 
\DIFdelbegin \DIFdel{very 
important }\DIFdelend \DIFaddbegin \DIFadd{significant }\DIFaddend role, such as the cusp conditions in the processes of 
electron capture and \DIFdelbegin \DIFdel{ionization}\DIFdelend \DIFaddbegin \DIFadd{ionisation}\DIFaddend . 

In previous works \cite{Mendez2015,Mendez2016,Mendez2018}, we 
developed the \DIFdelbegin \DIFdel{Depurated Inversion Method 
}\DIFdelend \DIFaddbegin \DIFadd{depurated inversion method }\DIFaddend (DIM), allowing to obtain 
accurate\DIFaddbegin \DIFadd{, }\DIFaddend effective potentials by substituting 
the coupled multielectron equations into a Kohn--Sham type equation. 
In \DIFdelbegin \DIFdel{a }\DIFdelend \DIFaddbegin \DIFadd{the }\DIFaddend first step, the corresponding effective potential is obtained 
through inversion of this equation. 
Next, \DIFdelbegin \DIFdel{a }\DIFdelend careful optimisation of this potential is carried on, eliminating 
poles, and imposing analytically, the appropriate boundary conditions. 
In that way, the DIM potentials are \DIFdelbegin \DIFdel{parametrized }\DIFdelend \DIFaddbegin \DIFadd{parametrised }\DIFaddend in simple analytical 
expressions.
In the present work, we show how these DIM potentials overcome the 
deficiencies of the pseudopotentials, calculating the cross sections 
for inelastic problems.

Finally, an extension of the DIM method for simple molecular \DIFdelbegin \DIFdel{system 
}\DIFdelend \DIFaddbegin \DIFadd{systems 
}\DIFaddend is developed, providing a new \DIFdelbegin \DIFdel{parameteric }\DIFdelend \DIFaddbegin \DIFadd{parametric }\DIFaddend expression and results for 
collisional processes. An implementation example is given for the 
methane molecule.


%%%%%%%%%%%%%%%%%%%%%%%%%%%%%%%%%%%%%%%%%%%%%%%%%%%%%%%%%%%%%%%%%%%%%
\section{Theory}

\subsection{Pseudopotentials}
\label{sec:PPs}

The pseudopotential (PP) method consists of replacing the Coulomb 
potential in the many--electron system Hamiltonian with a smooth 
function so that the electron wavefunctions oscillating rapidly in 
the core region are replaced by nodeless pseudo--orbitals having the 
\DIFdelbegin \DIFdel{correct }\DIFdelend \DIFaddbegin \DIFadd{right }\DIFaddend energy and the same outer range properties.
In general, the pseudopotentials $V_{\mbox{\tiny PP}}$ can be defined 
through a pseudo--charge $Z_{\mbox{\tiny PP}}$ as
\begin{equation}
 V_{\mbox{\tiny PP}}(r)=-\frac{Z_{\mbox{\tiny PP}}(r)}{r} \,,
 \label{eq:pseudopotential}
\end{equation}
\begin{equation}
 Z_{\mbox{\tiny PP}}(r)=\bigg\{
 \begin{array}{cl}
  f(r), & r\leq r_c \\
  1, & r>r_c 
 \end{array} 
 \,,
 \label{eq:pseudocharge}
\end{equation}
where $r_c$ is a cutoff radius that separates the core, $r\leq r_c$,
from the valence region, $r>r_c$, of the target and $f(r)$ is a 
\DIFdelbegin \DIFdel{continous }\DIFdelend \DIFaddbegin \DIFadd{continuous }\DIFaddend function with a constant value at the origin.
Fig.~\ref{fig:pseudoLi} illustrates a pseudopotential (solid line)
and its corresponding pseudo--wavefunction for the $3s$ orbital of 
argon. The one--electron Hartree--Fock wavefunction is included
with a dashed line. The pseudo--wavefunction agrees with the 
HF orbital only in the outer region, \DIFdelbegin \DIFdel{loosing all 
infomation }\DIFdelend \DIFaddbegin \DIFadd{losing all 
information }\DIFaddend about the atomic structure close to the origin.
\begin{figure}[H]
\centering
 \includegraphics[height=0.23\textheight]{figures/pseudopot/pseudopotAr.eps}
\hspace{0.2cm}
 \includegraphics[height=0.23\textheight]{figures/pseudopot/pseudofuncAr.eps}
 \caption{(a) Pseudopotential, (b) pseudo--wavefunction and 
 HF orbital for the $3s$ orbital of argon.}
 \label{fig:pseudoLi}
\end{figure}

In Section~\ref{sec:colproc}, we \DIFdelbegin \DIFdel{analyze }\DIFdelend \DIFaddbegin \DIFadd{analyse }\DIFaddend the feasibility of 
implementing PPs in collisional processes calculations for two 
simple atomic \DIFdelbegin \DIFdel{cases}\DIFdelend \DIFaddbegin \DIFadd{targets}\DIFaddend : hydrogen and lithium. For each atom, the 
following pseudopotentials are examined
\begin{equation}
 \begin{array}{clll}
  \mbox{Name} & \mbox{Source} & \mbox{Type} & \mbox{Ref.} \\
  \hline  
  A & \mbox{\sc abinit} & \mbox{GGA} & \cite{abinitwebpage,Hamann1979} \\
  P & \mbox{\sc parsec} & \mbox{Troullier Martins} & \cite{parsecwebpage,Chelikowsky1994}\,.
 \end{array}
 \label{eq:pseudosources}
\end{equation}
Although there is no electronic core for hydrogen orbitals, it
can still be readily described by the pseudopotential 
approximation. \DIFdelbegin \DIFdel{In fact, the }\DIFdelend \DIFaddbegin \DIFadd{The }\DIFaddend hydrogen pseudopotentials from 
(\ref{eq:pseudosources}) reproduce the exact orbitals very 
accurately. 
%DIF > !!!!!!!!!!
\DIFaddbegin \DIFadd{\textcolor{blue}{\bf Ref2: A physical interpretation on the PP is missing.}
%DIF > !!!!!!!!!!
}\DIFaddend 

We will now proceed to examine \DIFdelbegin \DIFdel{closely }\DIFdelend the pseudo--charges and its 
one--electron solutions for the lithium atom \DIFaddbegin \DIFadd{closely}\DIFaddend . 
First, we study the spatial and momentum representation of the 
pseudo--charges. The momentum--space equivalent of $Z(r)$ is given 
by the Fourier transform
\begin{equation}
 \widetilde{Z}(k) = \frac{1}{\sqrt{2\pi}} \int_{-\infty}^{+\infty} 
 Z(r)\,e^{-ikr}\,dr\,.
\end{equation}
The pseudo--charges from~(\ref{eq:pseudosources}) for the $2s$ 
orbital of lithium are illustrated in Fig.~\ref{fig:ZLi}. 
For comparison, we include the potential attained from implementing 
the \DIFdelbegin \DIFdel{Depurated Inversion Method }\DIFdelend \DIFaddbegin \DIFadd{depurated inversion method }\DIFaddend described in Section~\ref{sec:DIM}.
The pseudo--charges vanish at the origin, avoiding the divergence 
of the Coulomb potential. However, this feature comes at a price: 
the pseudo--charges in the spatial representation are repulsive 
around \mbox{$r= $1 a.u.}\DIFaddbegin \DIFadd{, }\DIFaddend and their momentum picture fails to represent the 
target for high $k$, showing an incorrect oscillatory behaviour for 
values greater than $k_c=(2\pi r_c)^{-1}\sim0.7$ a.u..
\begin{figure}[H]
\centering
 \includegraphics[height=0.23\textheight]{figures/pseudopot/ZLi.eps}
\hspace{0.2cm}
 \includegraphics[height=0.23\textheight]{figures/pseudopot/ZLik.eps}
 \caption{Pseudo and DIM charges for the $2s$ orbital of lithium. 
 (a) Spatial and (b) momentum representation.}
 \label{fig:ZLi}
\end{figure}

Secondly, we inspect the behaviour of the bound pseudo--orbitals 
obtained from solving the one--electron \DIFdelbegin \DIFdel{Schrodinger }\DIFdelend \DIFaddbegin \DIFadd{Schr\"odinger }\DIFaddend equation with a 
pseudopotential. As usual, the bound state wavefunctions can be 
written as
\begin{equation}
 \psi_{nlm}(\mathbf{r}) = \frac{u_{nl}(r)}{r}Y_l^m(\hat{r})\,,
 \label{eq:centralfield-wave}
\end{equation}
where $u_{nl}(r)$ are the reduced radial wavefunctions\DIFaddbegin \DIFadd{, }\DIFaddend and 
$Y_l^m(\hat{r})$ are \DIFdelbegin \DIFdel{th }\DIFdelend \DIFaddbegin \DIFadd{the }\DIFaddend spherical harmonics. Similarly, the Fourier 
transform of these functions \DIFdelbegin \DIFdel{are }\DIFdelend \DIFaddbegin \DIFadd{is }\DIFaddend given by
\begin{equation}
 \widetilde{\psi}_{nlm}(\mathbf{k}) =
 \frac{\chi_{nl}(k)}{k}Y_l^m(\hat{k})\,.
\end{equation}

The spatial and momentum \DIFdelbegin \DIFdel{representation }\DIFdelend \DIFaddbegin \DIFadd{representations }\DIFaddend of the $2s$ radial 
pseudo--wavefunctions of lithium corresponding to the pseudo--charges 
from~(\ref{eq:pseudosources}) are displayed in Fig.~\ref{fig:wavesLi}.
Although the pseudo--orbitals are very different
\DIFdelbegin \DIFdel{than }\DIFdelend \DIFaddbegin \DIFadd{from }\DIFaddend the HF $2s$ wavefunction, the transformed $\chi(k)$ seems to 
have similar characteristics. However, a closer inspection of the 
tail region of these functions (see the inset of the figure) 
\DIFdelbegin \DIFdel{show }\DIFdelend \DIFaddbegin \DIFadd{shows }\DIFaddend the existence of several nodes.
We will see later that these discrepancies introduce significant 
consequences in the cross sections for most of the collisional 
processes examined.
\begin{figure}[H]
\centering
 \includegraphics[height=0.23\textheight]{figures/pseudopot/funcLi.eps}
\hspace{0.2cm}
 \includegraphics[height=0.23\textheight]{figures/pseudopot/funckLi.eps}
 \caption{Pseudo and DIM bound state wavefunction in (a) spatial
 and (b) momentum representation for the
 $2s$ orbital of lithium.}
 \label{fig:wavesLi}
\end{figure}

Finally, the pseudopotential approach not only affects the 
representation of the bound orbitals \DIFdelbegin \DIFdel{, }\DIFdelend but also determines the form of 
the continuum wavefunctions. For large $r$, the free state orbitals of 
an electron in \DIFaddbegin \DIFadd{the }\DIFaddend presence of a Coulomb potential can be written as
\begin{equation}
 u_{kl}(r) \rightarrow \,\sin \left( kr - l\frac{\pi}{2} - \eta \ln 2kr +
 \sigma_l + \delta_l \right)\,,
\end{equation}
where $k$ is the particle wave number, $\eta$ is Sommerfeld's 
parameter, $\sigma_l$ is the Coulomb phase shift and $\delta_l$ is the 
wave phase shift with respect to the Coulomb wave.

Comparisons between the DIM (solid line) and the pseudo (dashed)
free $ks$ wavefunctions for lithium are shown in Fig.~\ref{fig:contLi},
close to the origin (left) and asymptotically (right).
The pseudo and DIM wavefunctions behave similarly far away from the nucleus.
The asymptotic phase shift $\Delta$ accounts for the differences 
between the potentials. As the 
energy of the free electron increases, $\Delta$ diminishes. 
However, the orbitals in the core region are different even with 
increasing energy; the first maximum of the DIM wavefunctions 
are consistently smaller than of the pseudo--orbitals, which is 
understood since the Coulomb--type attraction of the nuclei is 
stronger than the pseudopotential in that region. 
\begin{figure}[H]
\centering
\includegraphics[height=0.23\textheight]{figures/pseudopot/LicontA.eps}
% \hspace{0.01cm}
\includegraphics[height=0.23\textheight]{figures/pseudopot/LicontB.eps}
 \caption{Continuum wavefunctions with energies $E$ near the origin 
 (left) and in the asymptotic region (right), calculated with the 
 DIM potential (solid line) and the {\sc abinit} pseudopotential (dashed 
 line). }
 \label{fig:contLi}
\end{figure}

%%%%%%%%%%%%%%%%%%%%%%%%%%%%%%%%%%%%%%%%%%%%%%%%%%%%%%%%%%%%%%%%%%%%%
\subsection{Depurated Inversion Method \DIFdelbegin \DIFdel{potentials}\DIFdelend \DIFaddbegin \DIFadd{Potentials}\DIFaddend }
\label{sec:DIM}

The \DIFdelbegin \DIFdel{Depurated Inversion Method }\DIFdelend \DIFaddbegin \DIFadd{depurated inversion method }\DIFaddend \cite{Mendez2015,Mendez2016,Mendez2018} 
consists of assuming that the many--electron atom orbitals can be 
represented by the solution of Kohn--Sham type equations, in which 
the $nl$ effective potentials are given by 
\begin{equation}
V_{nl}(r) = 
\frac{1}{2}\frac{1}{u_{nl}(r)}
\frac{d^2u_{nl}(r)}{dr^{2}} - 
\frac{l(l+1)}{2r^{2}}+\varepsilon_{nl} \, ,
\label{eq:V}
\end{equation}
where $u_{nl}$ and $\varepsilon_{nl}$ are the $nl$ orbital wavefunctions and
energies, respectively. In this work, the atomic structure is
approximated with the Hartree--Fock method, 
which is computed with the {\sc hf} codes by 
C. \DIFdelbegin \DIFdel{F. }\DIFdelend \DIFaddbegin \DIFadd{Froese }\DIFaddend Fischer \cite{FroeseFischer1997} and the {\sc nrhf} code by 
W. Johnson \cite{Johnson2007}. The computation of Eq.~(\ref{eq:V}) 
poses various numerical problems. The nodes and asymptotic decay
of the wavefunctions $u_{nl}(r)$ introduce significant numerical errors in
the inversion procedure (see Ref.~\cite{Mendez2018} for further 
details). The nodes of the orbitals \DIFdelbegin \DIFdel{produces }\DIFdelend \DIFaddbegin \DIFadd{produce }\DIFaddend huge unphysical 
poles, while the rapid asymptotic decay of the internal wavefunctions 
generates large divergences in the tail region of the potentials.
The Depuration method is implemented to tackle these unphysical features. 
An effective potential with a Coulomb--type shape $V_r(r)=-Z_r(r)/r$ 
is defined, and we enforce the correct boundary 
conditions fitting the inverted potential with the following analytical 
expression
\begin{equation}
 Z_r(r) = \sum_{j=1}^n z_j e^{-\alpha_j r}(1+\beta_jr) + 1
 \, \longrightarrow \, \bigg\{
 \begin{array}{cc}
  Z_N, & r\rightarrow0 \\
  1, & r\rightarrow\infty 
 \end{array}
 \label{eq:atomiczDIM}
\end{equation} 
where \mbox{$\sum z_j = Z_N-1$}. Afterwards, the parameters are 
optimised to reproduce the HF values accurately. 

%%%%%%%%%%%%%%%%%%%%%%%%%%%%%%%%%%%%%%%%%%%%%%%%%%%%%%%%%%%%%%%%%%%%%
\section{Collisional \DIFdelbegin \DIFdel{processes }\DIFdelend \DIFaddbegin \DIFadd{Processes }\DIFaddend in \DIFdelbegin \DIFdel{atoms}\DIFdelend \DIFaddbegin \DIFadd{Atoms}\DIFaddend }
\label{sec:colproc}

The most \DIFdelbegin \DIFdel{significative }\DIFdelend \DIFaddbegin \DIFadd{significant }\DIFaddend advantage of the pseudopotential method is 
its \DIFdelbegin \DIFdel{symplicity}\DIFdelend \DIFaddbegin \DIFadd{simplicity}\DIFaddend . However, it is worth to determine the validity
of this approach for computing collisional processes.
In this Section, we perform a thorough examination of the
pseudo--potentials for hydrogen and 
lithium by comparing the cross sections of four inelastic 
processes: proton--impact excitation, proton--impact ionisation,
charge exchange and photoionisation. The initial and final states
of the targets are obtained by solving the corresponding Schr\"odinger 
equation.
For the hydrogen atom, we compare the pseudopotential results with 
the exact analytical solutions.  
Furthermore, in order to asses the \DIFdelbegin \DIFdel{aplicability of the 
Depurated Inversion Method}\DIFdelend \DIFaddbegin \DIFadd{applicability of the 
depurated inversion method}\DIFaddend , we compute the photoionisation of 
more complex many--electron atoms and 
compare our findings with experimental data.


%%%%%%%%%%%%%%%%%%%%%%%%%%%%%%%%%%%%%%%%%%%%%%%%%%%%%%%%%%%%%%%%%%%%%
\subsection{Proton--Impact Excitation}

The proton--impact excitation of target $X$ is defined as 
\begin{equation}
 \mbox{H}^+ + X \rightarrow \mbox{H}^+ + X^*\,.
\end{equation}
The excitation cross section $\sigma$ of the target from the initial bound
state $\psi_i$ to the excited state $\psi_{\!f}$ may be written as 
\begin{equation}
 \sigma=\frac{\mu^2}{4\pi^2}\frac{k_{\!f}}{k_i}\int{\left|T_{\!fi}\right|^2 
 d\Omega}\,,
  \label{eq:cross-section}
\end{equation}
where $\mu$ is the reduced mass of the proton--atom system, 
$\mathbf{k}_i$ and $\mathbf{k}_{\!f}$ are the initial and final relative 
\DIFdelbegin \DIFdel{momentum}\DIFdelend \DIFaddbegin \DIFadd{momenta}\DIFaddend , and 
\begin{equation}
 T_{\!fi}=\langle \psi_{\!f}|V|\psi_i\rangle
 \label{eq:transition-matrix}
\end{equation}
is the transition matrix or T--matrix.
If the initial and final \DIFdelbegin \DIFdel{state }\DIFdelend \DIFaddbegin \DIFadd{states }\DIFaddend of the transition are described by
the Hartree--Fock method, the orbitals will give the correct high 
energy limit in the first order approximation (this is not the case 
for the charge exchange process). Hence, we will concentrate our 
computing efforts in the first perturbative order of the transition
matrix element through the first Born (FB) approximation, given by
\begin{equation}
 T_{\!fi}^{\mbox{\scriptsize FB}} = \widetilde{V}(\mathbf{p}) F_{\!fi}(\mathbf{p}) \,.
 \label{eq:firstbornTmatrix}
\end{equation}
The term $F_{\!fi}(\mathbf{p})$ is the form--factor
\begin{equation}
 F_{fi}(\mathbf{p}) = \frac{1}{(2\pi)^{3/2}} 
 \int{\widetilde{\psi}_{\!f}^*(\mathbf{k})
 \widetilde{\psi}_i(\mathbf{k}+\mathbf{p})\,d\mathbf{k}}\,,
 \label{eq:form-factor}
\end{equation}
where $\mathbf{p}$ is the momentum transfer vector
\begin{equation}
 \mathbf{p} = p_{\mbox{\tiny min}} \hat{\mathbf{v}} + \boldsymbol{\eta} \,,
\end{equation}
\begin{equation}
 p_{\mbox{\tiny min}} = \frac{\varepsilon_{\!f}-\varepsilon_i}{v} \rightarrow 
 \bigg\{
 \begin{array}{cl}
  \infty, &v\rightarrow 0 \\
  0, &v\rightarrow \infty
 \end{array}
 \,,
 \label{eq:pmin}
\end{equation}
$\hat{\mathbf{v}}$ is the ion velocity, $\boldsymbol{\eta}$ is
the transversal momentum transfer, so that 
$\hat{\mathbf{v}} \cdot \boldsymbol{\eta}=0$, whereas $\varepsilon_i$ 
and $\varepsilon_{\!f}$ are the binding energies corresponding to the 
initial and final \DIFdelbegin \DIFdel{state}\DIFdelend \DIFaddbegin \DIFadd{states}\DIFaddend .

The first Born proton--impact excitation cross sections of hydrogen 
and lithium from the ground states are shown in Fig.~\ref{fig:exHLi}.
The pseudopotential results for the $n=2$ and $n=3$ final states of
hydrogen agree excellently with the analytical expression. For 
lithium, the pseudopotential cross sections agree in a broad velocity 
range with the DIM calculations, except for low proton--impact velocities. 
This disagreement arises from the form 
factor. For low impact velocities, the momentum transfer vector is large 
(\ref{eq:pmin}). As discussed earlier, the bound momentum orbital
$\widetilde{\psi}(\mathbf{k}+\mathbf{p})$ is not \DIFdelbegin \DIFdel{properly described }\DIFdelend \DIFaddbegin \DIFadd{described adequately 
}\DIFaddend by the pseudopotentials at this region. 
An alternative expression for the form factor can
be considered by implementing the peaking approximation
\begin{equation}
 F_{\!fi}(\mathbf{p}) \sim \widetilde{\psi}_i(\mathbf{p})\widetilde{\psi}_{\!f}^*(0)
 +\widetilde{\psi}_{\!f}(\mathbf{p})\widetilde{\psi}_i^*(0)\,.
\end{equation}
Therefore, in order to have the correct form factor\DIFaddbegin \DIFadd{, }\DIFaddend it is necessary to 
obtain an accurate description of the initial bound state at large
momentum values, which is not the case for the pseudostates (see 
Fig.~\ref{fig:wavesLi}b) and hence their failure.

\begin{figure}[H]
\centering
 \includegraphics[height=0.23\textheight]{figures/pseudopot/Hex.eps}
 \hspace{0.4cm}
 \includegraphics[height=0.23\textheight]{figures/pseudopot/Liex.eps}
 \caption{Proton--impact excitation cross section from the ground state 
 for hydrogen and lithium.}
 \label{fig:exHLi}
\end{figure}


%%%%%%%%%%%%%%%%%%%%%%%%%%%%%%%%%%%%%%%%%%%%%%%%%%%%%%%%%%%%%%%%%%%%%
\subsection{Proton--Impact Ionisation}

The transition matrix (\ref{eq:transition-matrix}) for the 
proton--impact ionisation of $X$,
\begin{equation}
 \mbox{H}^+ + X \rightarrow \mbox{H}^+ + X^+ + e^-\,,
\end{equation}
can also be written in terms of the first order Born
approximation. In this case, the final state $\psi_{\!f}$ in 
Eq.~(\ref{eq:form-factor}) is an outgoing continuum 
wavefunction $\psi_{\mathbf{k}_{\!f}}^-$\DIFdelbegin \DIFdel{and }\DIFdelend \DIFaddbegin \DIFadd{, while 
}\DIFaddend $\varepsilon_{\!f}=k_{\!f}^2/2$ \DIFaddbegin \DIFadd{is the energy of the ionised electron}\DIFaddend . 

The single--differential proton--impact ionisation cross sections 
$d\sigma/d\varepsilon_{\!f}$ of hydrogen and lithium at a proton 
velocity of $v_p=1$\,a.u. 
are shown in Fig.~\ref{fig:ionHLi}. 
In the case of hydrogen, the pseudopotential and analytical results
agree for all the electron energy range, except at very high values. 
On the other hand, for lithium, the cross sections computed with 
pseudopotentials only agree at low energies. 
Once again, assuming that $\psi_{\mathbf{k}_{\!f}}^-(\mathbf{k})$ can be 
approximated by a plane wave, the form factor is reduced to the 
Fourier transform of the initial bound state
\begin{equation}
 F_{\!fi}(\mathbf{p})\sim\widetilde{\psi}_i(\mathbf{p}-\mathbf{k}_{\!f})\,.
\end{equation}
Then, as $k_{\!f}$ increases, so does $p_{\mbox{\tiny min}}$, and the form
factor is not well represented by the pseudopotentials. The \DIFdelbegin \DIFdel{big 
}\DIFdelend \DIFaddbegin \DIFadd{significant 
}\DIFaddend discrepancies shown in Fig. \ref{fig:ionHLi} provides another demonstration
of how a wrong description of the momentum space wavefunction may
produce huge errors in collisional processes calculations.
\begin{figure}[H]
\centering
 \includegraphics[height=0.225\textheight]{figures/pseudopot/ionizationH.eps}
 \hspace{0.2cm}
 \includegraphics[height=0.225\textheight]{figures/pseudopot/ionizationLi.eps}
 \caption{Single differential proton--impact ionisation cross 
 section for hydrogen and lithium at \mbox{$v_p=1$\, a.u..}}
 \label{fig:ionHLi}
\end{figure}

%%%%%%%%%%%%%%%%%%%%%%%%%%%%%%%%%%%%%%%%%%%%%%%%%%%%%%%%%%%%%%%%%%%%%
\subsection{Proton--Impact Charge Exchange}

The proton--impact charge exchange of target $X$ is defined as
\begin{equation}
 \mbox{H}^+ + X \rightarrow \mbox{H} + X^+\,.
\end{equation}
The charge transfer cross section by the collision of a proton 
(electron capture) is 
computed with the first order Brinkman--Kramers 
approximation~\cite{Brinkman1930}. 
Accordingly, the matrix element is defined by 
\begin{equation}
 T_{\!fi}^{\mbox{\tiny BK}} = \widetilde{\psi}_f^*(\mathbf{W}_{\!f})
 \left[\varepsilon_{\!f}-\frac{W_{\!f}^2}{2} \right]
 \widetilde{\psi}_{\!i}(\mathbf{W}_i)\,,
\end{equation}
where $\mathbf{W}_{\!i}$ and $\mathbf{W}_{\!f}$ are the momentum transfer 
vectors 
\begin{equation}
 \mathbf{W}_{\!i} = W_{\!i0} \hat{\mathbf{v}} + \boldsymbol{\eta}, \quad 
 W_{\!i0} = \frac{v}{2} - p_{\mbox{\tiny min}}
 \label{eq:momtransfi0}
\end{equation}
\begin{equation}
 \mathbf{W}_{\!f} = W_{\!f0} \hat{\mathbf{v}} + \boldsymbol{\eta}, \quad 
 W_{\!f0} = \frac{v}{2} + p_{\mbox{\tiny min}}\,,
 \label{eq:momtransff0}
\end{equation}
and they satisfy the condition $\mathbf{W}_{\!i}+\mathbf{W}_{\!f}=\mathbf{v}$,
and $p_{\mbox{\tiny min}}$ is defined in Eq.~(\ref{eq:pmin}).

The charge exchange cross sections of hydrogen and lithium in the 
ground state are illustrated in Fig.~\ref{fig:captureH}. The cross 
section of hydrogen is described with high accuracy by the 
pseudopotential approach for a wide range of proton velocities. 
However, this process constitutes 
a symmetrical resonance, i.e. $\varepsilon_{\!f}=\varepsilon_{\!i}$, 
and the agreement may be misleading. For the lithium case, the 
pseudopotentials fail utterly to describe the electron capture 
correctly at low and high velocities. For low and high $v_p$ values, the 
momentum transfer vector becomes large\DIFaddbegin \DIFadd{, }\DIFaddend and therefore, the cross 
sections are wrongly calculated with pseudopotentials. 
The results disagree completely for most of the energy values, except
for a \DIFdelbegin \DIFdel{very small }\DIFdelend \DIFaddbegin \DIFadd{minimal }\DIFaddend range of velocities around $v\simeq0.77$\,a.u..
\begin{figure}[H]
\centering
 \includegraphics[height=0.23\textheight]{figures/pseudopot/captureH.eps}
 \hspace{0.3cm}
 \includegraphics[height=0.23\textheight]{figures/pseudopot/captureLi.eps}
 \caption{Proton--impact electron capture cross section for hydrogen and 
 lithium.}
 \label{fig:captureH}
\end{figure}

%%%%%%%%%%%%%%%%%%%%%%%%%%%%%%%%%%%%%%%%%%%%%%%%%%%%%%%%%%%%%%%%%%%%%
\subsection{Photoionisation}
\label{sec:photo}

The single photoionisation is defined as
\begin{eqnarray}
 \hbar\omega + X &\rightarrow& X^+ + e\,.
\end{eqnarray}
Considering a perturbative photon field, the initial bound $\psi_{i}$ 
and final continuum $\psi_{\mathbf{k}_{\!f}}^-$ states of the target are
not considerably distorted; therefore, the relevant matrix element of the 
photoionisation process is given by
\begin{equation}
 T_{\mathbf{k}}^{\mbox{\tiny Ph}} = \int{
 \psi_{\mathbf{k}_{\!f}}^-(\mathbf{r}) 
 \left[-i \hat{\boldsymbol{\varepsilon}}_\lambda \cdot 
 \boldsymbol{\nabla}_\mathbf{r}\right] 
 \psi_{i}(\mathbf{r})}\,,
\end{equation}
where $\hat{\boldsymbol{\varepsilon}}_\lambda$ is the polarisation 
versor and $\mathbf{k}_{\!f}=\sqrt{2(\omega+\varepsilon_i)}$, as imposed 
by \DIFdelbegin \DIFdel{the }\DIFdelend energy conservation.

The first--order photoionisation cross sections of hydrogen and 
lithium are shown in Fig.~\ref{fig:photoHLi}. The pseudopotentials 
results for the hydrogen atom agree with the exact analytical 
expression results only for low photon energies, failing at larger values. 
\DIFdelbegin \DIFdel{This }\DIFdelend \DIFaddbegin \DIFadd{These }\DIFaddend discrepancies can be understood considering the continuum 
wavefunction $\psi_{\mathbf{k}_{\!f}}^-(\mathbf{r})$ as a plane wave. 
Consequently, the matrix element $T_{\mathbf{k}}^{\mbox{\tiny Ph}}$ 
is reduced to
\begin{equation}
 T_{\mathbf{k}}^{\mbox{\tiny Ph}} \sim 
 -\left(\hat{\boldsymbol{\varepsilon}}_\lambda \cdot \mathbf{k}_{\!f} \right)
 \widetilde{\psi_i} \left(\mathbf{k}_{\!f}\right)\,,
\end{equation}
and it is determined entirely by the behaviour of the bound target 
pseudostate in the momentum representation. For hydrogen, the 
pseudo--orbital from {\sc parsec} in the Fourier space coincides with 
the exact 
analytical solution for the entire range of $k$, which explains the 
excellent agreement in the cross section results.
For lithium, the pseudopotential cross sections disagree with the DIM 
results for all energy values. The large \DIFdelbegin \DIFdel{oscilations }\DIFdelend \DIFaddbegin \DIFadd{oscillations }\DIFaddend in the cross sections
are originated by the spurious oscillatory structure of the bound state
for large  $k$ values (see inset of Fig.~\ref{fig:wavesLi}b).

\begin{figure}[H]
\centering
 \includegraphics[height=0.23\textheight]{figures/pseudopot/photoionH.eps}
 \hspace{0.2cm}
 \includegraphics[height=0.23\textheight]{figures/pseudopot/photoionLi.eps}
\caption{Single photoionisation cross section for hydrogen 
and lithium.}
 \label{fig:photoHLi}
\end{figure}

%%%%%%%%%%%%%%%%%%%%%%%%%%%%%%%%%%%%%%%%%%%%%%%%%%%%%%%%%%%%%%%%%%%%%
\subsection{DIM \DIFdelbegin \DIFdel{photoionisation }\DIFdelend \DIFaddbegin \DIFadd{Photoionisation }\DIFaddend of \DIFdelbegin \DIFdel{many--electron atoms}\DIFdelend \DIFaddbegin \DIFadd{Many--electron Atoms}\DIFaddend }

In order to asses the applicability of the \DIFdelbegin \DIFdel{Depurated Inversion Method
}\DIFdelend \DIFaddbegin \DIFadd{depurated inversion method
}\DIFaddend for atoms with a more complex structure, we compute the 
photoionisation of many--electron targets with the DIM 
potentials~\cite{Mendez2016} and compare our results with 
experimental values. 
The first order photoionisation cross section of nitrogen and neon 
are shown in Fig.~\ref{fig:photoDIM}. Experimental data 
from~\cite{Henke1993,Samson1990,Samson2002,Stolte2016} is illustrated 
with hollow symbols. The DIM photoionisation cross section
of these atoms agree excellently with the experimental values for 
low, medium and high photon--energies. For neon, discrepancies start 
to be noticeable for low and intermediate energy. An accurate 
photoionisation description of heavier atoms requires the inclusion 
of many--body effects that can be relevant, such as orbital 
relaxation due to the creation of a hole, collective response 
of inner shell electrons~\cite{Ederer1964} and correlation effects.

\begin{figure}[H]
\centering
 \includegraphics[height=0.23\textheight]{figures/dimpot/photoN.eps}
 \hspace{0.2cm}
 \includegraphics[height=0.23\textheight]{figures/dimpot/photoNe.eps}
\caption{Single photoionisation cross section for nitrogen and 
neon.}
\label{fig:photoDIM}
\end{figure}


%%%%%%%%%%%%%%%%%%%%%%%%%%%%%%%%%%%%%%%%%%%%%%%%%%%%%%%%%%%%%%%%%%%%%
\section{Depurated Inversion Method for Molecules}

The description of molecular systems constitutes a real challenge due 
to their \DIFdelbegin \DIFdel{generally multicentered and highly non--central nature}\DIFdelend \DIFaddbegin \DIFadd{nonspherical symmetry and multicenter character}\DIFaddend . 
Many \textit{ab initio} and semi--empirical theoretical approximations 
have been developed to this end over the last century.

In this work, the \DIFdelbegin \DIFdel{Depurated Inversion Method }\DIFdelend \DIFaddbegin \DIFadd{depurated inversion method }\DIFaddend is extended to determine 
effective potentials for molecules. In this Section, the theoretical 
methods are \DIFdelbegin \DIFdel{stablished }\DIFdelend \DIFaddbegin \DIFadd{established, }\DIFaddend and results for methane are given. Furthermore, 
collisional processes are computed with these molecular DIM potentials.

\subsection{Theory}

Without loss of generality, we will present the DIM theoretical grounds 
for hydride compounds. The Hamiltonian of \DIFdelbegin \DIFdel{a }\DIFdelend \DIFaddbegin \DIFadd{an }\DIFaddend $N$--electron $X\!H_n$ 
molecule within the Born--Oppenheimer approximation is given by
\begin{equation}
 \mathcal{H}=\DIFaddbegin \DIFadd{-}\DIFaddend \sum_{i=1}^N \frac{1}{2} \nabla^2_{\mathbf{r}_i} 
 - \sum_{i=1}^N \frac{Z_N}{r_i} 
 + \sum_{i=1}^N V_H(r_i)
 + \sum\DIFdelbegin \DIFdel{_{i=1}}\DIFdelend \DIFaddbegin \DIFadd{_{i<1}}\DIFaddend ^N \frac{1}{r_{ij}}\DIFaddbegin \,\DIFadd{,
}\DIFaddend \end{equation}
\begin{equation}
 V_H(r_i) = -\sum_{j=1}^{n} \frac{1}{\left|\mathbf{r}_i-\mathbf{R}_{H_j}\right|}\DIFaddbegin \,\DIFadd{,
}\DIFaddend \end{equation}
where $Z_N$ is the nuclear charge of the heavier atom\DIFaddbegin \DIFadd{, }\DIFaddend and 
$\mathbf{R}_{H_j}$ are the coordinates of the hydrogens \DIFdelbegin \DIFdel{with }\DIFdelend respect to 
the $X$ atom. The corresponding Schr\"odinger equation 
$\mathcal{H}\Psi=E\Psi$ is solved and the orbitals are expressed as
Eq.~(\ref{eq:centralfield-wave}) %DIF <  $\Psi_{nlm}=u_{nl}(r)/r\,Y_l^m(\hat{r})$
\DIFdelbegin \DIFdel{considering the single--center 
}\DIFdelend \DIFaddbegin \DIFadd{considering the single--centre
}\DIFaddend expansion (SCE). The orbitals and energies are found by solving the 
Hartree--Fock equations. The computation of these 
equations generally relies on the use of finite basis sets for the 
representation of the molecular orbitals (MOs). Usually, the MOs are 
expressed as a linear combination of atomic orbitals (LCAO),
\begin{equation}
 \Psi_i=\sum_j c_{ji} \phi_j\,,
\end{equation}
which can be constructed with Gaussian--type orbitals (GTO) basis 
sets. 

The inverted molecular potential expression, analogous 
to Eq.~(\ref{eq:V}), obtained from GTO basis sets present more difficulties
than the atomic case. In addition to the asymptotic divergences and
the poles, large unphysical 
oscillations arise~\cite{Schipper1997,Mura1997,Jacob2011,Gaiduk2013}. 
These \DIFdelbegin \DIFdel{big }\DIFdelend \DIFaddbegin \DIFadd{prominent }\DIFaddend oscillations are originated from imperceptible undulations 
present in the MOs due to the finite number of the basis set. The
second derivative, necessary to evaluate the \DIFdelbegin \DIFdel{invertion }\DIFdelend \DIFaddbegin \DIFadd{inversion }\DIFaddend formula, amplifies
these features~\cite{Schipper1997,Gaiduk2013}.
The appearance of these oscillations in the inverted potentials forces us to
incorporate further actions in the depuration scheme. To illustrate 
this procedure, we consider the $1s$ orbital of the carbon atom. 
We solved the Hartree--Fock equations using the \mbox{6-311G} basis set 
with {\sc gamess} code~\cite{Schmidt1993,Gordon2005}.
We obtain \DIFdelbegin \DIFdel{the }\DIFdelend inverted potentials by implementing Eq.~(\ref{eq:V}). 
The resulting $Z_{1s}^{\mbox{\scriptsize 6-311G}}$ charge is shown in 
Fig.~\ref{fig:1sCarbon}a with a dot--dashed line. 
The charge oscillates significantly at low distances and diverges for 
higher $r$ values. The same calculation was 
repeated using the universal Gaussian basis set (UGBS), which has a 
more significant amount of primitives. The corresponding inverted 
charge $Z_{1s}^{\mbox{\scriptsize UGBS}}$ is exhibited in the figure 
with a dashed line. 
Although the charge still diverges around $r\approx1\,$a.u., the 
oscillations are now circumscribed near the nucleus.
Finally, the differential Hartree--Fock equations for the carbon 
atom were solved using the finite--differences (FD) method. 
The $1s$ inverted charge obtained with this procedure, 
$Z_{nl}^{\mbox{\scriptsize FD}}$, shows no oscillations since no basis 
sets have 
been used to construct the orbital; however, the charge still 
diverges for $r>1\,$a.u., as it usually does for all HF calculations.

\begin{figure}[H]
\centering
\includegraphics[height=0.23\textheight]{figures/dimpot/1sCGMSS_BS.eps}
\hspace{0.3cm}
\includegraphics[height=0.23\textheight]{figures/dimpot/1sC_oscprof.eps}
\caption{(a) Effective charges for the $1s$ orbital of carbon. 
(b) Basis--set oscillation profiles.}
\label{fig:1sCarbon}
\end{figure}

As shown in the figure, for each basis set used in the calculations,
different oscillation profiles will arise, which are defined as 
\begin{equation}
 p_{nl}^{\mbox{\scriptsize BS}} = Z_{nl}^{\mbox{\scriptsize BS}}-
 Z_{nl}^{\mbox{\scriptsize FD}} \,,
 \label{eq:oscillation-prof}
\end{equation}
where $Z_{nl}^{\mbox{\scriptsize BS}}$ is the inverted charge of the atom 
using the particular basis set ``BS'' and 
$Z_{nl}^{\mbox{\scriptsize FD}}$ is the effective charge obtained 
from the \DIFdelbegin \DIFdel{invertion }\DIFdelend \DIFaddbegin \DIFadd{inversion }\DIFaddend of the finite--difference wavefunctions. 
In the previous example, the basis set considered for calculating 
the $1s$ orbital of carbon were \mbox{6-311G} and UGBS. The 
oscillation profiles for the $1s$ orbital, using Eq.~(\ref{eq:oscillation-prof}) 
for these basis sets, are shown in Fig.~\ref{fig:1sCarbon}b. 
Since the orbital profiles for each atomic basis set are distinctive,
once they are determined for the atomic case, they can be removed 
in further molecular calculations. An example of this procedure 
is given in the following Section.

%%%%%%%%%%%%%%%%%%%%%%%%%%%%%%%%%%%%%%%%%%%%%%%%%%%%%%%%%%%%%%%%%%%%%
\subsection{Example: Methane}
\label{sec:dimmethane}

In order to illustrate the implementation of the DIM for molecules, 
we considered the methane molecule, which has a suitable
spherical geometry. We computed the HF molecular orbitals and 
energies of CH$_4$ employing the UGBS basis sets of carbon and 
hydrogen, using the {\sc gamess} code. 
The charges obtained by direct inversion are given in 
Fig.~\ref{fig:ch4zeff} with dashed lines. 
Since the molecular orbitals are given by LCAO of carbon and 
hydrogen, the oscillations of the inverted charges are a 
consequence of the finite basis set of these atoms. 
To remove the most \DIFdelbegin \DIFdel{important oscillationsit is necessary 
to }\DIFdelend \DIFaddbegin \DIFadd{critical oscillations,  first, we must 
}\DIFaddend determine the oscillation profiles produced by the atomic carbon basis set.
We use Eq.~(\ref{eq:oscillation-prof}) to determine the 
$p_{1s}^{\mbox{\scriptsize UGBS}}$, $p_{2s}^{\mbox{\scriptsize UGBS}}$ 
and $p_{2p}^{\mbox{\scriptsize UGBS}}$ profiles of carbon. 
Then, we remove the oscillations by 
\DIFdelbegin \DIFdel{substracting }\DIFdelend \DIFaddbegin \DIFadd{subtracting }\DIFaddend the carbon $p_{nl}^{\mbox{\scriptsize UGBS}}$ profiles 
to the corresponding inverted charges 
$Z_{i}^{\mbox{\scriptsize UGBS}}$ of CH$_4$.
The oscillations are removed for all orbitals except for the 
$2a_2$, which presents small oscillatory residues from the 
hydrogen basis set. Since the residual fluctuations are \DIFdelbegin \DIFdel{very small
}\DIFdelend \DIFaddbegin \DIFadd{minimum
}\DIFaddend and near the nucleus, we proceeded to implement the depuration 
scheme as described in Section~\ref{sec:DIM}. We define a new 
parametric DIM charge equation,
\begin{eqnarray}
 Z(r) = \sum_j Z_j e^{-\alpha_j r} 
 + Z_{\mbox{\scriptsize H}} e^{-(\ln r - \ln \beta)^2/(2\gamma)} 
 + 1\,.
 \label{eq:moleculezDIM}
\end{eqnarray}
In contrast to the approximation proposed for 
atoms~(\ref{eq:atomiczDIM}), a second term has been added to the 
formula to account for the presence of the hydrogens.
The optimised parameters for the methane molecule are given in 
Table~\ref{tab:ch4parameters}\DIFaddbegin \DIFadd{, }\DIFaddend and the corresponding DIM 
charges are shown in Fig.~\ref{fig:ch4zeff}, with solid lines. 
The orbital energies obtained with these charges are also given in 
the table.

\begin{figure}[t]
\centering
\includegraphics[height=0.23\textheight]{figures/dimpot/zeff_1sCH4.eps}
\hspace{0.3cm}
\includegraphics[height=0.23\textheight]{figures/dimpot/zeff_2sCH4.eps}

\vspace{0.25cm}
\includegraphics[height=0.23\textheight]{figures/dimpot/zeff_2pCH4.eps}
\caption{Effective charges of CH$_4$; 
direct inversion (dashed line) and depurated inverted (solid line).}
\label{fig:ch4zeff}
\end{figure}

\begin{table}[H]
\centering
\begin{tabular}{cc|ccccc}
\hline
   $nl$ & $E$ &$Z$ & $\alpha$ & $\beta$ & $\gamma$ \\
\hline
   $1a_1$ & -11.1949  & 1.925280 & 0.641982 & & \\
          & & 0.953120 & 5.571510 & & \\
          & & 2.121600 & 1.500440 & & \\
   $2a_2$ & -0.9204 & 2.912200 & 3.149990 & & \\
          & & 2.087800 & 0.771371 & & \\
          & & 1.23640  &          & 2.329570 & 0.053420 \\
   $2t_1$ & -0.5042 & 0.901953 & 2.895140 & & \\
          & & 1.112030 & 0.388649 & & \\
          & & 2.986017 & 2.931210 & & \\
          & & 1.30182  &          & 2.169850 & 0.012616 \\
\hline
\end{tabular}
\caption{Energies and fitting parameters for the DIM effective charges 
(Eq.~(\ref{eq:moleculezDIM})), for CH$_4$.}
\label{tab:ch4parameters}
\end{table}


%%%%%%%%%%%%%%%%%%%%%%%%%%%%%%%%%%%%%%%%%%%%%%%%%%%%%%%%%%%%%%%%%%%%%
\subsection{Collisional \DIFdelbegin \DIFdel{processes}\DIFdelend \DIFaddbegin \DIFadd{Processes}\DIFaddend }

The orientation of the molecular targets \DIFdelbegin \DIFdel{are important }\DIFdelend \DIFaddbegin \DIFadd{is crucial }\DIFaddend for determining
the cross sections of collisional processes. However, the molecular 
orientations in most collisional experiments are generally not 
pre--established. Thus, the \DIFdelbegin \DIFdel{spherical }\DIFdelend \DIFaddbegin \DIFadd{spherically }\DIFaddend averaged description of the system, 
assumed by the DIM potential is a valid approximation. In the 
following, we examine two collisional processes in the first--order 
approximation: proton--impact ionisation 
and single photoionisation. 

\subsubsection{Proton--Impact Ionisation}

Results for the proton--impact ionisation cross section for CH$_4$,
calculated under the first Born approximation, are given in 
Fig.~\ref{fig:ionch4}. 
Experimental data from Ref.~\cite{Rudd1983,Rudd1985} are displayed 
with symbols. The initial bound and the final continuum states 
of the molecule needed for the T--matrix computation 
(Eq.~(\ref{eq:transition-matrix})) were calculated 
with the DIM potentials from Section~\ref{sec:dimmethane}. 
The photoionisation cross section for high and intermediate energies
shows good agreement with the experimental results. The failure at
low energies is attributed to the validity of the first Born 
approximation and not to our DIM approach.

\begin{figure}[H]
\centering
\includegraphics[height=0.23\textheight]{figures/dimpot/born_ionch4.eps}
\caption{Proton--impact ionisation cross section for CH$_4$. Solid 
line: first--order DIM theoretical calculations. Symbols: experiments 
from Ref.~\cite{Rudd1983,Rudd1985}.}
\label{fig:ionch4}
\end{figure}

\subsubsection{Photoionisation}
\label{subsubsec:photoionisation}

The photoionisation cross section for CH$_4$, calculated with the DIM 
potentials in a first order approximation, is shown in 
Fig.~\ref{fig:photoch4} (solid lines).
Good agreement with the experimental results (symbols) is found for 
high energy values and at the thresholds. The curve between $\sim$15 and
$\sim$300 eV shows the photoionisation from the outer $n=2$ shell\DIFdelbegin \DIFdel{. 
The pronounced }\DIFdelend \DIFaddbegin \DIFadd{, while 
the }\DIFaddend discontinuity at 300 eV corresponds to the threshold of
the $1a_1$ inner shell orbital. For low and intermediate photon--energies,
the agreement between our calculations and the experimental values 
from Ref.~\cite{Lukirskii1964,Henke1982,Samson1989} is not perfect.
Phenomena such as molecular orbital relaxation, possible 
collective contributions and correlation effects must be considered 
in further calculations. On the other hand, for the $1a_1$ inner shell
photoionisation, these effects are not significant\DIFaddbegin \DIFadd{, }\DIFaddend and we obtain a 
perfect agreement with the experimental results.
\begin{figure}[H]
\centering
\includegraphics[height=0.23\textheight]{figures/dimpot/photoch4.eps}
\caption{Single photoionisation cross section of CH$_4$.
Solid line: first--order DIM theoretical calculations. Symbols: experiments
from Ref.~\cite{Lukirskii1964,Henke1982,Samson1989}.}
\label{fig:photoch4}
\end{figure}


\section{Concluding remarks}

In this work, we explored the possibility of using pseudopotentials 
within the single electron model to calculate inelastic transitions. 
The first Born approximation was used to calculate proton--impact 
excitation, \DIFdelbegin \DIFdel{ionization}\DIFdelend \DIFaddbegin \DIFadd{ionisation}\DIFaddend , electron capture and \DIFdelbegin \DIFdel{photoionization}\DIFdelend \DIFaddbegin \DIFadd{photoionisation}\DIFaddend . Two simple
atoms were studied, having a single electron in the outer shell. For 
hydrogen, we found \DIFdelbegin \DIFdel{great }\DIFdelend \DIFaddbegin \DIFadd{excellent }\DIFaddend agreement for all the collisional processes, for 
low and intermediate energies. In the case of lithium, the only process
that can be calculated with \DIFdelbegin \DIFdel{a }\DIFdelend reasonable accuracy is the proton--impact
excitation. We concluded that the range of validity is restrained to \DIFdelbegin \DIFdel{very
small }\DIFdelend \DIFaddbegin \DIFadd{minimal
}\DIFaddend momentum transfers. 
The Depurated Inversion method, on the other hand, accurately reproduce
photoionisation experimental results for many--electron atoms. 

We extended the DIM for molecular systems. In this case, the inversion
procedure raises huge oscillation due \DIFaddbegin \DIFadd{to }\DIFaddend the finite size of the basis \DIFaddbegin \DIFadd{sets
}\DIFaddend involved 
in the Hartree--Fock orbital calculations. An additional step is included 
during the depuration scheme. In order to determine the oscillation profile
for a particular basis set, we computed the \DIFdelbegin \DIFdel{atomic inverted }\DIFdelend \DIFaddbegin \DIFadd{inverted atomic }\DIFaddend charges also in
a finite--differences framework. By \DIFdelbegin \DIFdel{substracting }\DIFdelend \DIFaddbegin \DIFadd{subtracting }\DIFaddend the charges, it is possible
to isolate the oscillations corresponding to this particular basis set.
We used the DIM method to determine the effective potentials for CH$_4$.
These potentials are implemented in \DIFdelbegin \DIFdel{a }\DIFdelend first--order proton--impact ionisation
and photoionisation cross sections calculations. For both processes, we
found \DIFdelbegin \DIFdel{very }\DIFdelend good agreement with the experimental results. The main
discrepancies can be attributed to the fact that only first--order is 
considered in the perturbation theory.


\section*{Acknowledgement}
The authors thank the Consejo Nacional de Investigaciones 
Cient\'ificas y T\'ecnicas (CONICET), Universidad de Buenos Aires (UBA)
and Agencia Nacional de Promoci\'on Cient\'ifica y Tecnol\'ogica (ANPCyT)
for the grants that supported this work.


%%%%%%%%%%%%%%%%%%%%%%%%%%%%%%%%%%%%%%%%%%%%%%%%%%%%%%%%%%%%%%%%%%%%%
%DIF <  \bibliographystyle{plain}
% \bibliography{dimmol}
%DIF >  \bibliographystyle{achemso}

\begin{thebibliography}{10}

\bibitem{parsecwebpage}
\DIFaddbegin \DIFadd{PARSEC Home Page. 
}\DIFaddend \url{https://parsec.ices.utexas.edu/styled-2/}\DIFaddbegin \DIFadd{,
(accessed Jan 18, 2019).
}\DIFaddend 

\bibitem{Chelikowsky1994}
\DIFaddbegin \DIFadd{Chelikowsky, }\DIFaddend J. R.\DIFdelbegin \DIFdel{Chelikowsky}\DIFdelend \DIFaddbegin \DIFadd{; Troullier}\DIFaddend , N.\DIFdelbegin \DIFdel{Troullier, and Y.
Saad,
}\DIFdelend \DIFaddbegin \DIFadd{; Saad, Y.
}\DIFaddend Finite--difference--pseudopotential method: Electronic structure calculations without a basis\DIFdelbegin \DIFdel{,
}\DIFdelend \DIFaddbegin \DIFadd{.
}{\it \DIFaddend Phys. Rev. Lett.\DIFaddbegin } \DIFaddend {\bf \DIFdelbegin \DIFdel{72}\DIFdelend \DIFaddbegin \DIFadd{1994}\DIFaddend }, \DIFaddbegin \DIFadd{72, }\DIFaddend 1240--1243\DIFdelbegin \DIFdel{(1994)}\DIFdelend .

\bibitem{Mendez2015}
\DIFaddbegin \DIFadd{Mendez, }\DIFaddend A. M. P.
\DIFdelbegin \DIFdel{Mendez,
}\DIFdelend M\'etodo de Inversi\'on Depurada para Potenciales Locales en \'Atomos y Mol\'eculas\DIFdelbegin \DIFdel{, 
}\DIFdelend \DIFaddbegin \DIFadd{.
}\DIFaddend Tesis de Licenciatura, Universidad Nacional de Salta\DIFdelbegin \DIFdel{(2015).
}\DIFdelend \DIFaddbegin \DIFadd{, April 2015.
}\DIFaddend 

\bibitem{Mendez2016}
\DIFaddbegin \DIFadd{Mendez, }\DIFaddend A. M. P.\DIFdelbegin \DIFdel{Mendez}\DIFdelend \DIFaddbegin \DIFadd{; Mitnik}\DIFaddend , D. M.\DIFdelbegin \DIFdel{Mitnik, and }\DIFdelend \DIFaddbegin \DIFadd{; Miraglia, }\DIFaddend J. E.
\DIFdelbegin \DIFdel{Miraglia,
}\DIFdelend Depurated inversion method for orbital--specific exchange potentials.
\DIFaddbegin {\it \DIFaddend Int. J. Quantum Chem.\DIFaddbegin } \DIFaddend {\bf \DIFdelbegin \DIFdel{116}\DIFdelend \DIFaddbegin \DIFadd{2016}\DIFaddend }, \DIFdelbegin \DIFdel{24 (2016)}\DIFdelend \DIFaddbegin \DIFadd{116, 1882--1890}\DIFaddend .

\bibitem{Mendez2018}
\DIFaddbegin \DIFadd{Mendez, }\DIFaddend A. M. P.\DIFdelbegin \DIFdel{Mendez}\DIFdelend \DIFaddbegin \DIFadd{; Mitnik}\DIFaddend , D. M.\DIFdelbegin \DIFdel{Mitnik, and }\DIFdelend \DIFaddbegin \DIFadd{; Miraglia, }\DIFaddend J. E. 
\DIFdelbegin \DIFdel{Miraglia,
}\DIFdelend Local Effective Hartree--Fock Potentials Obtained by the Depurated Inversion Method\DIFdelbegin \DIFdel{, 
}\DIFdelend \DIFaddbegin \DIFadd{. 
In }{\it \DIFadd{Nov. Electron. Struct. Theory Gen. Innov. Strongly Correl. Syst.}}\DIFadd{; 
Hoggan, P. E., Ed.; 
}\DIFaddend Advances in Quantum Chemistry\DIFdelbegin %DIFDELCMD < {\bf %%%
\DIFdelend \DIFaddbegin \DIFadd{;
Academic Press, 2018;
Vol.~}\DIFaddend 76\DIFdelbegin %DIFDELCMD < }%%%
\DIFdel{, }\DIFdelend \DIFaddbegin \DIFadd{; pp~}\DIFaddend 117--132\DIFdelbegin \DIFdel{(2018)}\DIFdelend .

\bibitem{abinitwebpage}
\DIFaddbegin \DIFadd{ABINIT Home Page. 
}\DIFaddend \url{https://www.abinit.org/psp-tables}\DIFaddbegin \DIFadd{,
(accessed Jan 18, 2019).
}\DIFaddend 

\bibitem{Hamann1979}
\DIFaddbegin \DIFadd{Hamann, }\DIFaddend D. R.\DIFdelbegin \DIFdel{Hamann}\DIFdelend \DIFaddbegin \DIFadd{; Schl\"uter}\DIFaddend , M.\DIFdelbegin \DIFdel{Schl\"uter, and C.
Chiang,
}\DIFdelend \DIFaddbegin \DIFadd{; Chiang, C.
}\DIFaddend Norm--Conserving Pseudopotentials\DIFdelbegin \DIFdel{,
}\DIFdelend \DIFaddbegin \DIFadd{.
}{\it \DIFaddend Phys. Rev. Lett.\DIFaddbegin } \DIFaddend {\bf \DIFdelbegin \DIFdel{43}\DIFdelend \DIFaddbegin \DIFadd{1979}\DIFaddend }, \DIFaddbegin \DIFadd{43, }\DIFaddend 1494--1497\DIFdelbegin \DIFdel{(1979)}\DIFdelend .

\bibitem{FroeseFischer1997}
\DIFdelbegin \DIFdel{C. }\DIFdelend Froese Fischer, \DIFaddbegin \DIFadd{C.; Brage, }\DIFaddend T.\DIFdelbegin \DIFdel{Brage, and P.}\DIFdelend \DIFaddbegin \DIFadd{; }\DIFaddend J\"onsson, \DIFaddbegin \DIFadd{P.
}\DIFaddend {\em Computational \DIFdelbegin \DIFdel{atomic structure: an MCHF approach}\DIFdelend \DIFaddbegin \DIFadd{Atomic Structure: An MCHF Approach}\DIFaddend },
Institute of Physics Publishing\DIFdelbegin \DIFdel{(1997).
}\DIFdelend \DIFaddbegin \DIFadd{: Bristol, UK, 1997.
}\DIFaddend 

\bibitem{Johnson2007}
\DIFaddbegin \DIFadd{Johnson, }\DIFaddend W. R. 
\DIFdelbegin \DIFdel{Johnson,
}\DIFdelend {\em Atomic \DIFdelbegin \DIFdel{structure theory : lectures on atomic physics}\DIFdelend \DIFaddbegin \DIFadd{Structure Theory: Lectures on Atomic Physics}\DIFaddend },
\DIFdelbegin \DIFdel{Springer (2007).
}\DIFdelend \DIFaddbegin \DIFadd{Springer--Verlag Berlin Heidelberg, 2007.
}\DIFaddend 

\bibitem{Brinkman1930}
\DIFdelbegin \DIFdel{Von }\DIFdelend \DIFaddbegin \DIFadd{Brinkman, }\DIFaddend H. C.\DIFdelbegin \DIFdel{Brinkman and }\DIFdelend \DIFaddbegin \DIFadd{; Kramers, }\DIFaddend H. A.
\DIFdelbegin \DIFdel{Kramers,
}\DIFdelend Zur Theorie der Einfangung von Elektronen durch $\alpha$--Teilchen.
\DIFaddbegin {\it \DIFaddend Proc. K. Akad. van Wet.\DIFaddbegin } \DIFaddend {\bf \DIFdelbegin \DIFdel{33}\DIFdelend \DIFaddbegin \DIFadd{1930}\DIFaddend }, \DIFaddbegin \DIFadd{33, }\DIFaddend 973--984\DIFdelbegin \DIFdel{(1930)}\DIFdelend .

\bibitem{Henke1993}
\DIFaddbegin \DIFadd{Henke, }\DIFaddend B. L.\DIFdelbegin \DIFdel{Henke}\DIFdelend \DIFaddbegin \DIFadd{; Gullikson}\DIFaddend , E. M.\DIFdelbegin \DIFdel{Gullikson, and }\DIFdelend \DIFaddbegin \DIFadd{; Davis, }\DIFaddend J. C. 
\DIFdelbegin \DIFdel{Davis,
}\DIFdelend X--Ray Interactions: Photoabsorption, Scattering, Transmission, and Reflection at $E$=50--30000 eV, $Z$=1--92\DIFdelbegin \DIFdel{,
}\DIFdelend \DIFaddbegin \DIFadd{.
}{\it \DIFaddend At. Data Nucl. Data Tables\DIFaddbegin } \DIFaddend {\bf \DIFdelbegin \DIFdel{54}\DIFdelend \DIFaddbegin \DIFadd{1993}\DIFaddend }, \DIFaddbegin \DIFadd{54, }\DIFaddend 181--342\DIFdelbegin \DIFdel{(1993)}\DIFdelend .

\bibitem{Samson1990}
\DIFaddbegin \DIFadd{Samson, }\DIFaddend J. A. R.\DIFdelbegin \DIFdel{Samson and }\DIFdelend \DIFaddbegin \DIFadd{; Angel, }\DIFaddend G. C.
\DIFdelbegin \DIFdel{Angel,
}\DIFdelend Single-- and double--photoionization cross sections of atomic nitrogen from threshold to 31 \AA\DIFdelbegin \DIFdel{,
}\DIFdelend \DIFaddbegin \DIFadd{.
}{\it \DIFaddend Phys. Rev. A\DIFaddbegin } \DIFaddend {\bf \DIFdelbegin \DIFdel{42}\DIFdelend \DIFaddbegin \DIFadd{1990}\DIFaddend }, \DIFaddbegin \DIFadd{42, }\DIFaddend 1307--1312\DIFdelbegin \DIFdel{(1990)}\DIFdelend .

\bibitem{Samson2002}
\DIFaddbegin \DIFadd{Samson, }\DIFaddend J. A. R.\DIFdelbegin \DIFdel{Samson and }\DIFdelend \DIFaddbegin \DIFadd{; Stolte, }\DIFaddend W. C.
\DIFdelbegin \DIFdel{Stolte,
}\DIFdelend Precision measurements of the total photoionization cross--sections of He, Ne, Ar, Kr, and Xe\DIFdelbegin \DIFdel{,
}\DIFdelend \DIFaddbegin \DIFadd{.
}{\it \DIFaddend J. Electron Spectros. Relat. Phenomena\DIFaddbegin } \DIFaddend {\bf \DIFdelbegin \DIFdel{123}\DIFdelend \DIFaddbegin \DIFadd{2002}\DIFaddend }, \DIFaddbegin \DIFadd{123, }\DIFaddend 265--276\DIFdelbegin \DIFdel{(2002)}\DIFdelend .

\bibitem{Stolte2016}
\DIFaddbegin \DIFadd{Stolte, }\DIFaddend W. C.\DIFdelbegin \DIFdel{Stolte}\DIFdelend \DIFaddbegin \DIFadd{; Jonauskas}\DIFaddend , V.\DIFdelbegin \DIFdel{Jonauskas}\DIFdelend \DIFaddbegin \DIFadd{; Lindle}\DIFaddend , D. W.\DIFdelbegin \DIFdel{Lindle}\DIFdelend \DIFaddbegin \DIFadd{; Sant'Anna}\DIFaddend , M. M.\DIFdelbegin \DIFdel{Sant'Anna, and }\DIFdelend \DIFaddbegin \DIFadd{; Savin, }\DIFaddend D. W. 
\DIFdelbegin \DIFdel{Savin,
}\DIFdelend Inner--shell Photoionization studies of neutral atomic nitrogen\DIFdelbegin \DIFdel{,
}\DIFdelend \DIFaddbegin \DIFadd{.
}{\it \DIFaddend Astrophys. J.\DIFaddbegin } \DIFaddend {\bf \DIFdelbegin \DIFdel{818}\DIFdelend \DIFaddbegin \DIFadd{2016}\DIFaddend }, \DIFdelbegin \DIFdel{149 (2016).
}\DIFdelend \DIFaddbegin \DIFadd{818, 149.
}\DIFaddend 

\bibitem{Ederer1964}
\DIFaddbegin \DIFadd{Ederer, }\DIFaddend D. L. 
\DIFdelbegin \DIFdel{Ederer,
}\DIFdelend Photoionization of the $4d$ electrons in Xenon\DIFdelbegin \DIFdel{,
}\DIFdelend \DIFaddbegin \DIFadd{.
}{\it \DIFaddend Phys. Rev. Lett.\DIFaddbegin } \DIFaddend {\bf \DIFdelbegin \DIFdel{13}\DIFdelend \DIFaddbegin \DIFadd{1964}\DIFaddend }, \DIFaddbegin \DIFadd{13, }\DIFaddend 760--762\DIFdelbegin \DIFdel{(1964)}\DIFdelend .

\bibitem{Schipper1997}
\DIFaddbegin \DIFadd{Schipper, }\DIFaddend P. R. T.\DIFdelbegin \DIFdel{Schipper}\DIFdelend \DIFaddbegin \DIFadd{; Gritsenko}\DIFaddend , O. V.\DIFdelbegin \DIFdel{Gritsenko, and }\DIFdelend \DIFaddbegin \DIFadd{; Baerends, }\DIFaddend E. J. 
\DIFdelbegin \DIFdel{Baerends,
}\DIFdelend Kohn-Sham potentials corresponding to Slater and Gaussian basis set densities\DIFdelbegin \DIFdel{,
}\DIFdelend \DIFaddbegin \DIFadd{.
}{\it \DIFaddend Theor. Chem. Accounts: Theory, Comput. Model. (Theoretica Chim. Acta)\DIFaddbegin } \DIFaddend {\bf \DIFdelbegin \DIFdel{98}\DIFdelend \DIFaddbegin \DIFadd{1997}\DIFaddend }, \DIFaddbegin \DIFadd{98, }\DIFaddend 16--24\DIFdelbegin \DIFdel{(1997)}\DIFdelend .

\bibitem{Mura1997}
\DIFaddbegin \DIFadd{Mura, }\DIFaddend M. E.\DIFdelbegin \DIFdel{Mura}\DIFdelend \DIFaddbegin \DIFadd{; Knowles}\DIFaddend , P. J.\DIFdelbegin \DIFdel{Knowles, and }\DIFdelend \DIFaddbegin \DIFadd{; Reynolds, }\DIFaddend C. A.
\DIFdelbegin \DIFdel{Reynolds,
}\DIFdelend Accurate numerical determination of Kohn--Sham potentials from electronic densities: I. Two--electron systems\DIFdelbegin \DIFdel{,
}\DIFdelend \DIFaddbegin \DIFadd{.
}{\it \DIFaddend J. Chem. Phys.\DIFaddbegin } \DIFaddend {\bf \DIFdelbegin \DIFdel{106}\DIFdelend \DIFaddbegin \DIFadd{1997}\DIFaddend }, \DIFaddbegin \DIFadd{106, }\DIFaddend 9659--9667\DIFdelbegin \DIFdel{(1997)}\DIFdelend .

\bibitem{Jacob2011}
\DIFaddbegin \DIFadd{Jacob, }\DIFaddend C. R. 
\DIFdelbegin \DIFdel{Jacob,
}\DIFdelend Unambiguous optimization of effective potentials in finite basis sets\DIFdelbegin \DIFdel{,
}\DIFdelend \DIFaddbegin \DIFadd{.
}{\it \DIFaddend J. Chem. Phys.\DIFaddbegin } \DIFaddend {\bf \DIFdelbegin \DIFdel{135}\DIFdelend \DIFaddbegin \DIFadd{2011}\DIFaddend }, \DIFdelbegin \DIFdel{244102 (2011).
}\DIFdelend \DIFaddbegin \DIFadd{135, 244102.
}\DIFaddend 

\bibitem{Gaiduk2013}
\DIFaddbegin \DIFadd{Gaiduk, }\DIFaddend A. P.\DIFdelbegin \DIFdel{Gaiduk}\DIFdelend \DIFaddbegin \DIFadd{; Ryabinkin}\DIFaddend , I. G.\DIFdelbegin \DIFdel{Ryabinkin, and }\DIFdelend \DIFaddbegin \DIFadd{; Staroverov, }\DIFaddend V. N.
\DIFdelbegin \DIFdel{Staroverov,
}\DIFdelend Removal of Basis--Set Artifacts in Kohn--Sham Potentials Recovered from Electron Densities\DIFdelbegin \DIFdel{,
}\DIFdelend \DIFaddbegin \DIFadd{.
}{\it \DIFaddend J. Chem. Theory Comput.\DIFaddbegin } \DIFaddend {\bf \DIFdelbegin \DIFdel{9}\DIFdelend \DIFaddbegin \DIFadd{2013}\DIFaddend }, \DIFaddbegin \DIFadd{9, }\DIFaddend 3959--3964\DIFdelbegin \DIFdel{(2013)}\DIFdelend .

\bibitem{Schmidt1993}
\DIFaddbegin \DIFadd{Schmidt, }\DIFaddend M. W.\DIFdelbegin \DIFdel{Schmidt}\DIFdelend \DIFaddbegin \DIFadd{; Baldridge}\DIFaddend , K. K.\DIFdelbegin \DIFdel{Baldridge}\DIFdelend \DIFaddbegin \DIFadd{; Boatz}\DIFaddend , J. A.\DIFdelbegin \DIFdel{Boatz}\DIFdelend \DIFaddbegin \DIFadd{; Elbert}\DIFaddend , S. T.\DIFdelbegin \DIFdel{Elbert}\DIFdelend \DIFaddbegin \DIFadd{; Gordon}\DIFaddend , M. S.\DIFdelbegin \DIFdel{Gordon}\DIFdelend \DIFaddbegin \DIFadd{; Jensen}\DIFaddend , J. H.\DIFdelbegin \DIFdel{Jensen}\DIFdelend \DIFaddbegin \DIFadd{; Koseki}\DIFaddend , S.\DIFdelbegin \DIFdel{Koseki}\DIFdelend \DIFaddbegin \DIFadd{;
Matsunaga}\DIFaddend , N.\DIFdelbegin \DIFdel{Matsunaga}\DIFdelend \DIFaddbegin \DIFadd{; Nguyen}\DIFaddend , K. A.\DIFdelbegin \DIFdel{Nguyen}\DIFdelend \DIFaddbegin \DIFadd{; Su}\DIFaddend , S.\DIFdelbegin \DIFdel{Su}\DIFdelend \DIFaddbegin \DIFadd{; Windus}\DIFaddend , T. L.\DIFdelbegin \DIFdel{Windus}\DIFdelend \DIFaddbegin \DIFadd{; Dupuis}\DIFaddend , M.\DIFdelbegin \DIFdel{Dupuis, and }\DIFdelend \DIFaddbegin \DIFadd{; Montgomery, }\DIFaddend J. A.
\DIFdelbegin \DIFdel{Montgomery,
}\DIFdelend General atomic and molecular electronic structure system\DIFdelbegin \DIFdel{,
}\DIFdelend \DIFaddbegin \DIFadd{.
}{\it \DIFaddend J. Comput. Chem.\DIFaddbegin } \DIFaddend {\bf \DIFdelbegin \DIFdel{14}\DIFdelend \DIFaddbegin \DIFadd{1993}\DIFaddend }, \DIFaddbegin \DIFadd{14, }\DIFaddend 1347--1363\DIFdelbegin \DIFdel{(1993)}\DIFdelend .

\bibitem{Gordon2005}
\DIFaddbegin \DIFadd{Gordon, }\DIFaddend M. S.\DIFdelbegin \DIFdel{Gordon and }\DIFdelend \DIFaddbegin \DIFadd{; Schmidt, }\DIFaddend M. W.
\DIFdelbegin \DIFdel{Schmidt,
}\DIFdelend Advances in electronic structure theory: GAMESS a decade later\DIFdelbegin \DIFdel{,
in }\DIFdelend \DIFaddbegin \DIFadd{. 
In }{\it \DIFaddend Theory Appl. Comput. Chem.\DIFdelbegin \DIFdel{, }%DIFDELCMD < {\bf %%%
\DIFdel{41}\DIFdelend }\DIFdelbegin \DIFdel{, 1167--1189 (}\DIFdelend \DIFaddbegin \DIFadd{; 
Dykstra, C. E.; Frenking, G.; Kim, K. S.; Scuseria, G. E. Eds;
Elsevier: Amsterdam, }\DIFaddend 2005\DIFdelbegin \DIFdel{)}\DIFdelend \DIFaddbegin \DIFadd{; pp 1167--1189}\DIFaddend .

\bibitem{Rudd1983}
\DIFaddbegin \DIFadd{Rudd, }\DIFaddend M. E.\DIFdelbegin \DIFdel{Rudd}\DIFdelend \DIFaddbegin \DIFadd{; DuBois}\DIFaddend , R. D.\DIFdelbegin \DIFdel{DuBois}\DIFdelend \DIFaddbegin \DIFadd{; Toburen}\DIFaddend , L. H.\DIFdelbegin \DIFdel{Toburen}\DIFdelend \DIFaddbegin \DIFadd{; Ratcliffe}\DIFaddend , C. A.\DIFdelbegin \DIFdel{Ratcliffe, and }\DIFdelend \DIFaddbegin \DIFadd{; Goffe, }\DIFaddend T. V.
\DIFdelbegin \DIFdel{Goffe,
}\DIFdelend Cross sections for ionization of gases by 5--4000 keV protons and for electron capture by 5--150 keV protons\DIFdelbegin \DIFdel{,
}\DIFdelend \DIFaddbegin \DIFadd{.
}{\it \DIFaddend Phys. Rev. A\DIFaddbegin } \DIFaddend {\bf \DIFdelbegin \DIFdel{28}\DIFdelend \DIFaddbegin \DIFadd{1983}\DIFaddend }, \DIFaddbegin \DIFadd{28, }\DIFaddend 3244--3257\DIFdelbegin \DIFdel{(1983)}\DIFdelend .

\bibitem{Rudd1985}
\DIFaddbegin \DIFadd{Rudd, }\DIFaddend M. E.\DIFdelbegin \DIFdel{Rudd}\DIFdelend \DIFaddbegin \DIFadd{; Kim}\DIFaddend , Y. K.\DIFdelbegin \DIFdel{Kim}\DIFdelend \DIFaddbegin \DIFadd{; Madison}\DIFaddend , D. H.\DIFdelbegin \DIFdel{Madison, and }\DIFdelend \DIFaddbegin \DIFadd{; Gallagher, }\DIFaddend J. W.
\DIFdelbegin \DIFdel{Gallagher,
}\DIFdelend Electron production in proton collisions: total cross sections\DIFdelbegin \DIFdel{,
}\DIFdelend \DIFaddbegin \DIFadd{.
}{\it \DIFaddend Rev. Mod. Phys.\DIFaddbegin } \DIFaddend {\bf \DIFdelbegin \DIFdel{57}\DIFdelend \DIFaddbegin \DIFadd{1985}\DIFaddend }, \DIFaddbegin \DIFadd{57, }\DIFaddend 965--994\DIFdelbegin \DIFdel{(1985)}\DIFdelend .

\bibitem{Lukirskii1964}
\DIFaddbegin \DIFadd{Lukirskii, }\DIFaddend A. P.\DIFdelbegin \DIFdel{Lukirskii}\DIFdelend \DIFaddbegin \DIFadd{; Brytov}\DIFaddend , I. A.\DIFdelbegin \DIFdel{Brytov, and }\DIFdelend \DIFaddbegin \DIFadd{; Zimkina, }\DIFaddend T. M.
\DIFdelbegin \DIFdel{Zimkina,
}\DIFdelend \DIFaddbegin {\it \DIFaddend Optika i spektr.\DIFaddbegin } \DIFaddend {\bf \DIFdelbegin \DIFdel{17}\DIFdelend \DIFaddbegin \DIFadd{1964}\DIFaddend }, \DIFdelbegin \DIFdel{234 (1964).
}\DIFdelend \DIFaddbegin \DIFadd{17, 234.
}\DIFaddend 

\bibitem{Henke1982}
\DIFaddbegin \DIFadd{Henke, }\DIFaddend B. L.\DIFdelbegin \DIFdel{Henke}\DIFdelend \DIFaddbegin \DIFadd{; Lee}\DIFaddend , P.\DIFdelbegin \DIFdel{Lee}\DIFdelend \DIFaddbegin \DIFadd{; Tanaka}\DIFaddend , T. J.\DIFdelbegin \DIFdel{Tanaka}\DIFdelend \DIFaddbegin \DIFadd{; Shimabukuro}\DIFaddend , R. L.\DIFdelbegin \DIFdel{Shimabukuro, and }\DIFdelend \DIFaddbegin \DIFadd{; Fujikawa, }\DIFaddend B. K.
\DIFdelbegin \DIFdel{Fujikawa,
}\DIFdelend Low--energy X--ray interaction coefficients: Photoabsorption, scattering, and reflection: $E$=100--2000 eV $Z$=1--94\DIFdelbegin \DIFdel{,
}\DIFdelend \DIFaddbegin \DIFadd{.
}{\it \DIFaddend At. Data Nucl. Data Tables\DIFaddbegin } \DIFaddend {\bf \DIFdelbegin \DIFdel{27}\DIFdelend \DIFaddbegin \DIFadd{1982}\DIFaddend }, \DIFaddbegin \DIFadd{27, }\DIFaddend 1--144\DIFdelbegin \DIFdel{(1982)}\DIFdelend .

\bibitem{Samson1989}
\DIFaddbegin \DIFadd{Samson, }\DIFaddend J. A. R.\DIFdelbegin \DIFdel{Samson}\DIFdelend \DIFaddbegin \DIFadd{; Haddad}\DIFaddend , G. N.\DIFdelbegin \DIFdel{Haddad}\DIFdelend \DIFaddbegin \DIFadd{; Masuoka}\DIFaddend , T.\DIFdelbegin \DIFdel{Masuoka}\DIFdelend \DIFaddbegin \DIFadd{; Pareek}\DIFaddend , P. N.\DIFdelbegin \DIFdel{Pareek, and }\DIFdelend \DIFaddbegin \DIFadd{; Kilcoyne, }\DIFaddend D. A. L.
\DIFdelbegin \DIFdel{Kilcoyne,
}\DIFdelend Ionization yields, total absorption, and dissociative photoionization cross sections of CH4 from 110 to 950 \AA\DIFdelbegin \DIFdel{,
}\DIFdelend \DIFaddbegin \DIFadd{.
}{\it \DIFaddend J. Chem. Phys.\DIFaddbegin } \DIFaddend {\bf \DIFdelbegin \DIFdel{90}\DIFdelend \DIFaddbegin \DIFadd{1989}\DIFaddend }, \DIFaddbegin \DIFadd{90, }\DIFaddend 6925--6932\DIFdelbegin \DIFdel{(1989)}\DIFdelend .

\end{thebibliography}

\end{document}
