cuando uno hace un reporte, empieza con una breve descripcion del paper.
El referee no te esta preguntando nada al principio...

No es verdad que Moccia no tenga la cusp correctamente.
Lo que no tiene es el correcto valor de la carga cuando la invertimos.

En el comentario 7, decile que si, que es correcto y que eso ya lo
mostro Staroverov (lo citamos en el paper?)

La 8 te esta diciendo justamente que hagas HF sin bases,
con Diferencias Finitas. Lo ue podes decir es que igualmente
vas a teneer los problemas de nodos que tuviste cuando hiciste
HF con DF en atomos (y le citas el trabajo anterior del DIM).

Por que le pones CETO a ETO?

En la 4 te pide physical interpretation a PP en H.
H has only one electron, and it does not seem to have any use to
calculate the corresponding pseudopotential.
However, there are many different proposals for PP, reproducing
with high accuracy the main features of the wavefunctions, even
for excited states.

En la pagina de pseudopontenciales de at dice:
Comment:
Some people consider H-pseudopotentials as a nonsense. Nevertheless 
 this PP gives excellent description of the bond length for the H$ _2$ 
 dimer, and for H on C surfaces, and it requires only 200 eV.


En la respuestas al otro referee, no se entiende lo que decis en la
respuesta 6 : "... we can minimize to the mean values of our choosing" ???
Pusiste potentia en lugar de potential.
Creo que le tenes que poner lo que dijimos:
The referee is correct. he molecular calculations ...
... (d functions). We are aware that including these functions produces
changes in the energies of about xxx as is shown in Refs [32,33].
However, our interest is to understand the effect of the basis set,
isolating them, and compute the atomic oscillations profiles as a way to
remedy the big fluctuations in the inverted charges.




En la 2: no se si es because the one-electron approximation is being  
implemented.
En realidad vos calculas N-electron wavefunctions, pero al ser una  
one-electron
transition, en el calculo de la T matrix solo te queda el orbital 
 inicial y final
que cambiaron, mientras que el resto es ortonormal y no aporta nada.

Podes poner que es because we are calculated one-electron transitions in
first order approximation, involving only the initial and final state 
 orbitals.



